\documentclass[12pt, a4paper, oneside]{ctexart}
\usepackage{amsmath, amsthm, amssymb, graphicx}
\usepackage[bookmarks=true, colorlinks, citecolor=blue, linkcolor=black]{hyperref}
\newtheorem{theorem}{}
\newtheorem{subtheorem}{}

\title{生物资源评估}
\author{Liiii00}

\begin{document}
\maketitle

\newpage
\section{公式}

\begin{theorem}Russell原理
    $$\underset
    {
        \text
        {
            $B_t$:t时的生物量 \quad 
            $R$:补充量 \quad 
            $G$:生长量 \quad 
            $M$:自然死亡量 \quad
            $Y$:捕捞死亡量
        }
    }
    {
        B_2=B_1+R+G-M-Y
    }
    $$
\end{theorem}

\begin{theorem}生长方程
    \begin{subtheorem}VBGF
        $$L_t = L_{\infty} \bigg(1-e^{-K(t-t_0)}\bigg)$$ 
        $$W_t = W_{\infty} \bigg(1-e^{-K(t-t_0)}\bigg)^b$$ 
    \end{subtheorem}
    
    \begin{subtheorem}Logistic
        $$L_t = \frac{L_{\infty}}{1+e^{\alpha-rt}}$$ 
        $$W_t = \frac{W_{\infty}}{\big(1+e^{\alpha-rt}\big)^b}$$ 
    \end{subtheorem}

    \begin{subtheorem}Gompertz
        $$L_t = L_{\infty}e^{-g\cdot(e^{-rt})}$$ 
        $$W_t = W_{\infty}e^{-bg\cdot{(e^{-rt})}}$$ 
    \end{subtheorem}
\end{theorem}

\begin{theorem}CPUE
    $$CPUE=U=\frac{C}{f}$$
    $$F=qf$$
    $$S(L)=\frac{1}{1+e^{(S1-S2L)}} \Rightarrow ln\big(\frac{1}{S(L)}-1\big) = S_1 - S_2L$$
\end{theorem}

\begin{theorem}死亡
    $$Z=F+M$$
    $$A+S=1$$
    $$N_{t2}=N_{t1}\cdot e^{-Z(t_2-t_1)}$$
    $$D_{t_1 \rightarrow t_2}=N_t \cdot \big(1-e^{-Z\Delta t}\big)$$
    $$C_{t_1 \rightarrow t_2}=\frac{F}{Z} \cdot N_{t1} \cdot \big( 1-e^{-Z \Delta t} \big)$$
    $$E=\frac{C}{D}=\frac{F}{Z}$$
    $$D=Z(t_2-t_1)\overline{N} $$

    总死亡系数估计:

    BH方法:
    $$Z=K\frac{L_{\infty}-\overline{L} }{\overline{L}-L_c}$$

    渔获量曲线法:
    $$C_{i+m}=UR \cdot e^{-Zm}$$
    $$ln(C_{i+m})=ln(UR) - Zm$$

    取对数后变为线性关系$Y=A+BX$,使用Excel中的INTERCEPT与SLOPE获取A,B的值即可

\end{theorem}

\begin{theorem}实际种群分析(F会变)

    $$C_8=\frac{F_8}{Z_8}N_8(1-e^{-Z_8})$$

    推出

    $$N_8=\frac{N_8Z_8}{F_8(1-e^{-Z_8})}$$

    联立
    $$N_8=N_7*e^{-Z_7}$$
    $$C_7=\frac{F_7}{Z_7}N_7(1-e^{-Z_7})$$

    得出预测的Ci
    $$C_{pre7}=\frac{F_7}{Z_7}N_8(e^{Z7}-1)$$

    对$C_7-C_{pre7}$进行规划求解(目标为0,变量为F7)即可求出N7
    
\end{theorem}

\begin{theorem}离散型YPR

    通过生长方程算$L_t$,通过$W=aL^b$算体重,通过渔获量方程算渔获量C与剩余量N,再有:

    $$Y_{wi}=C_i \cdot W_i$$
    或
    $$Y_{wi}=C_i \cdot (W_i+W_{i+1})$$

    $$Y_{w}/R=\sum \frac{Y_{wi}}{R}$$

    通过规划求解调整F使$Y_{w}/R$最大即可
\end{theorem}

\begin{theorem}剩余产量模型
    $$B_{t+1}=B_t+rB_t\big(1-\frac{B_t}{K}\big)-C_t$$
    平衡状态下:
    $$U_{tpre}=qB_t=q(B_{t-1}+rB_{t-1}(1-\frac{B_{t-1}}{K})-C_{tobs})$$
    $$U_{obs}=\frac{C_{obs}}{f_{obs}}$$
    $$RSS=\sum (U_{tobs}-U_{tpre})^2$$
    使用规划求解求出使RSS的最小的k,q,r即可
    
    $$B_{MSY}=\frac{K}{2}$$
    $$MSY=\frac{rK}{4}$$
    $$F_{MSY}=\frac{r}{2}$$
    $$f_{MSY}=\frac{F_{MSY}}{q}$$
\end{theorem}

YPR摆了,不看了......
\end{document}
